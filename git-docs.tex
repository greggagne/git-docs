% XeLaTeX can use any Mac OS X font. See the setromanfont command below.
% Input to XeLaTeX is full Unicode, so Unicode characters can be typed directly into the source.

% The next lines tell TeXShop to typeset with xelatex, and to open and save the source with Unicode encoding.

%!TEX TS-program = xelatex
%!TEX encoding = UTF-8 Unicode

\documentclass[11pt]{article}
\usepackage{geometry}                % See geometry.pdf to learn the layout options. There are lots.
\geometry{letterpaper}                   % ... or a4paper or a5paper or ... 
%\geometry{landscape}                % Activate for for rotated page geometry
%\usepackage[parfill]{parskip}    % Activate to begin paragraphs with an empty line rather than an indent
\usepackage{graphicx}
\usepackage{amssymb}
\usepackage{hyperref}

\usepackage{verbatim}

\usepackage{fancyhdr}

\pagestyle{fancy}

\lhead{An Overview of Using {\tt git}}

\rhead{\today}


% Will Robertson's fontspec.sty can be used to simplify font choices.
% To experiment, open /Applications/Font Book to examine the fonts provided on Mac OS X,
% and change "Hoefler Text" to any of these choices.

\usepackage{fontspec,xltxtra,xunicode}
\defaultfontfeatures{Mapping=tex-text}
\setromanfont[Mapping=tex-text]{Hoefler Text}
\setsansfont[Scale=MatchLowercase,Mapping=tex-text]{Gill Sans}
\setmonofont[Scale=MatchLowercase]{Andale Mono}

\title{An Overview of Using {\tt Git}}
\author{Westminster College Computer Science Department}
%\date{}                                           % Activate to display a given date or no date

\begin{document}
\maketitle

\newpage

\noindent
{\bf Overview} \\

\noindent
The Git revision control system allows one to maintain different versions of files, create separate branches for experimentation, and restore to previous versions. It is an excellent tool for  managing software projects, both individually as well as collaborating with a group. It can be combined with GitHub which provides a remote repository where several users can collaborate on a shared code base and synchronize and coordinate their activities. \\

\noindent
This document provides an overview of a few simple git commands and concepts; online documentation provides more thorough coverage of this rich toolset. \\

\noindent
{\bf How to Create a Repository} \\

\noindent
Make a directory where your repository will be stored. \\

{\tt mkdir git-repo} \\

\noindent
and now {\tt cd} to that directory. \\ 

\noindent
Next, initialize the {\tt git} repository: \\

{\tt git init} \\

\noindent
To determine the status of your repository, enter: \\

{\tt git status} \\

\noindent
Now, you must populate the repository with some files. Create the following 3 files in the {\tt git-repo} directory: \\

{\tt file1.txt} \\ 

{\tt file2.txt} \\ 

{\tt file3.txt} \\ 

\noindent
{\bf How to Commit to the Repository} \\

\noindent
{\tt git} distinguishes between {\bf staged} and {\bf unstaged} files. Only staged files can be committed to the repository. To stage a file, you must add it to the repository: \\

{\tt git add file1.txt file2.txt file3.txt} \\

\noindent
This could have been simplified with the command \\

{\tt git add -A} \\

\noindent
which adds all files to the staging. \\

\noindent
Next, you must commit the contents of the staging area to the repository: \\

{\tt git commit -m "Initial Commit"} \\

\noindent
The string following the {\tt -m} option allows you to enter a string message describing the changes you are committing. In this instance, since it is only the initial commit, we will provide the appropriate message. \\

\noindent
We may not want all files sent to the staging area and committed to the repository. In this instance, we can use a {\tt .gitignore} file which specifies which files we {\bf do not} want committed to the repository. \\

\noindent
The following command can be used to determine the status of the repository since the most recent commit: \\

{\tt git status} \\

\noindent
{\bf Branching and Merging} \\

\noindent
A git repository has a single {\bf master} branch. From this master branch, you may create separate branches to try different features. The basic approach is that your master branch contains your working code, any separate {\bf developer} branches contain features you are experimenting with. When code you are experimenting with has passed tests, you can then integrate -- or {\bf merge} -- the development branch. \\

\noindent
To create a branch named {\tt dev} , enter the command \\

{\tt git branch dev} \\

\noindent 
which creates the branch named {\tt dev} . Next you must switch to that using using the {\tt checkout}
command: \\

{\tt git checkout dev} \\

\noindent
This switches your branch to the {\tt dev} branch. Now, any work you perform in the {\tt dev} branch is independent from the master branch. The {\tt checkout} command allows you to switch between any branches you wish. \\

\noindent
You can also merge your {\tt dev} branch with the master by entering \\

{\tt git checkout master} 

{\tt git merge dev} \\

\noindent
This first switches you back to the master branch, and then merges any changes in the {\tt dev} branch with the master branch. \\

\noindent
It is possible the master and {\tt dev} branches have inconsistent changes, and if git is unable to merge the two branches, the files containing conflicts will have messages that must be resolved. Once the conflicts have been addressed, the changes must be committed. \\

\noindent
{\bf Restoring Previous Commits}\\

\noindent
Each {\tt git} commit associates an {\small MD5} checksum with the transaction. This checksum allows you to revert to previous instances from the repository. \\

\noindent
The following command outputs the log history of all commits, including the checksum associated with each commit: \\

{\tt git log} \\


\noindent
If you wish to restore the state to the most recent previous commit (let's assume you had made several changes to files and wish to revert to the previous commit), enter: \\

{\tt git reset ----hard [commit]}\\

\noindent
This restores the branch to the specified commit. For example, if the checksum for a commit you wish to revert to is {\tt e9e4794b1218c312d44e7b3b16367ec510ae150a}, you would enter \\

\noindent
If you wish to restore to earlier commits, you would enter \\

{\tt git reset ----hard e9e4794b1218c312d44e7b3b16367ec510ae150a} \\

\noindent
The {\tt reset} command can also be used by specifying which version of {\tt HEAD} to reset to. The value of {\tt HEAD} is the most recent commit, {\tt HEAD\^{}} the second-most, etc. The following examples illustrate: \\

{\tt git reset ----hard HEAD\^{}} // one commit back \\ 

{\tt git reset ----hard HEAD\~{}3} // two commits back \\

\noindent
and so forth. \\
  
\noindent
{\bf Working with a Remote Repository} \\

\noindent
A {\tt git} repository can be deployed as a remote repository. The steps outlined in the section describe using \href{http://github.com}{github} as a remote repository. This requires  having a github account and the repository has been constructed. Follow the instructions for creating a repository on github. This repository will initially be empty. You will then push (i.e. move) a local repository to github.\\

\noindent
Let's assume the name of that repository on github is {\tt git-repo} and is associated with the user {\tt charliebrown} . The {\small URL} for this github repository appears as \\

{\tt https://github.com/charliebrown/git-repo.git} \\

\noindent
Initially, you will create a local repository and commit changes to it outlined in the steps for creating a repository and committing to it. Once you have committed changes to your repository, you can then push the repository to github according to the following steps: \\

{\tt git remote add origin https://github.com/charliebrown/git-repo.git} \\

\noindent
This specifies the {\small URL} name of the repository as the remote origin. \\

\noindent
Now, you must {\bf push} your local repo to github: \\

{\tt git push origin master} \\

\noindent
This sends your commits in your local repository to github. \\

\noindent
Another collaborator can then {\bf clone} the repository by entering \\

{\tt git clone https://github.com/charliebrown/git-repo.git} \\

\noindent
If one makes changes to the contents of their local repository, they must add the changed file to the staging area, followed by committing the change. Changes are then pushed to the remote repository. \\

\noindent
Once someone has cloned a remote repository, they can then get the latest changes by {\bf pulling} from the repository: \\

{\tt git pull origin master} \\

\noindent
After cloning a repository, a good practice is to always first pull from that remote repository before making changes. This ensures you are always working with the latest updates from the remote repository. \\


\end{document}  